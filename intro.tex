Every day a large number of events are generated in the mobile telecommunications networks. These mobile network events include calls, internet traffic usage, billing information, regular pings from and to mobile phones. This data is often referred to as Call Detail Record (CDR).

Telecommunication providers want to exploit CDR data for analyzing their customer's behavior.They are interested in questions such as:
\begin{itemize}
\item How many people visit this region?
\item How much time do they stay?
\item How often do they come?
\item Where do they live and work?
\item What are their demographics?
\end{itemize}

Smartphones leave users with more digital footprints than ever before. Telecommunications providers can collect CDR data with additional socio-demographic information such as age, gender, education level, income and residency gaining comprehensive insight into customer behaviour. These insights can not only be used by companies for providing geo-localized advertisements for their customers, but can also benefit society as a whole. For example, understanding human mobility for epidemic control, urban planning, traffic monitoring and forecasting.

Due to increasing focus on privacy and data protection, the CDR events are fully anonymized, the personal references are removed, even the timestamp is blurred. The users can only have unique identification for short time-frame, such as 24 hours. It is therefore challenging to obtain information such as returning customers in an area. It is a reasonable requirement for telecommunications providers to efficiently utilize the collected data and at the same time preserve the anonymity of its customers. 

Most of the use cases for analyzing CDR data require extracting some knowledge about the location or the trajectory of the customer groups. For telecommunication providers Global Positioning System (GPS) data is usually not available. Therefore we assume that a customer's route can be approximated by the trajectory of the mobile network events generated by them. The location of a mobile network event can be determined from a cell tower data-set, based on the cell id assigned to the event.

For this assumption to hold, it is very important to know how accurate the approximation is. In other words, we want to evaluate how well we can guess the true path of a customer from their mobile network events by determining the distance of these pairs of trajectories. 

Processing CDR data is not straightforward and it can present challenges such as temporal and spatial sparseness. Also the different networks from different providers do not always provide the same information. There can be different fields in different formats, as there is no standardization implemented yet. 

If we can validate the assumption that the customer's trajectory can be approximated by the trajectory of the mobile network events generated by them, we can apply the trajectory data mining techniques and methods already used for GPS trajectory analysis. 

Since the size of CDR data sets can be extremely large, it can no longer be processed by a single machine efficiently. Therefore it is of high importance to have the proper processing infrastructure in place.  

Due to the fact that big data technologies are emerging, it is possible to efficiently process large data sets for different use cases in the telecommunications domain like network anomaly detection, customer behavior analysis, traffic monitoring and so on.

In this project, a set of trajectories from mobile network events, a set of GPS trajectories and a large set of cell tower data from Berlin region is processed. 


