The main goal is to assess how well can the customer's location be approximated given the mobile network events they generate.

We examine how accurately we can estimate a customer's position given
\begin{itemize}
    \item sequence of mobile network events generated by them (CDR data),
    \item cell tower data (cell map) to obtain location information for the connected cells.
\end{itemize}

For validation we use the trajectory of GPS locations, which are considered as ground truth.

As the users tend to move while using their phones, the generated CDR observations have semantics of trajectories. CDR data points can be grouped as trajectories so that the characteristics of trajectories and the result of previous research already conducted in this field can be utilized for achieving more accurate positioning.

\newdef{definition}{Definition}
\begin{definition}
A point $P(x,y)$ can be defined as a pair of latitude $y$ and longitude $x$ values representing a geographical location.
\end{definition}

A trajectory reflects the motion history of a moving object. In practice trajectory is not continuous due to the limited sampling capability, therefore one reasonable approach is to consider trajectories as a sequence of points.

Evaluating cellular network event based customer positioning boils down to defining and calculating a distance or (dis)similarity measure between point based trajectories.

\begin{definition}
A trajectory $T$ with length $n$ is defined as a time-stamped sequence of its consecutive points: \[T={(t_{1},P_{1}), (t_{2},P_{2}), .. (t_{n},P_{n}}), T \in M\]
where $M$ is the set of possible trajectories.
\end{definition}

In this paper we do not attempt to reconstruct trajectories from CDR data considering the road network of Berlin. Instead we consider the sequence of positioned mobile network events together with the GPS observation and we try to determine the distance of the two trajectories based on the distances of individual pairs of points.

Similarly to the formulation in \cite{encyclopedia} and \cite{distance-def} we define distance measure, such as:

\begin{definition}
Let $\mathcal{M}$ be a set of trajectories. A function $d :\mathcal{M} \times \mathcal{M} \rightarrow \mathcal{R}$ is called a dissimilarity (distance) on $\mathcal{M}$ if for all $T_{1}, T_{2} \in \mathcal{M}$: 
\begin{itemize}
    \item $D(T_{1},T_{2}) \geqslant 0$
    \item $D(T_{1},T_{2}) = d(T_{2},T_{2})$
    \item $D(T_{1},T_{1}) = 0$
\end{itemize}
If all of these conditions are satisfied and $D(T_{1}, T_{2}) = 0 \Rightarrow  T_{1} = T_{2} $ is considered to be a symmetric. If
the triangle inequality is also satisfied, $D$ is a metric.
\end{definition}





