\documentclass{sig-alternate-05-2015}
%Hide ISBN from bottom
\makeatletter
\def\@copyrightspace{\relax}
\makeatother

\usepackage{subfiles}
\usepackage{nomencl}
\makenomenclature
\usepackage{hyperref}
\usepackage{footnote}
\usepackage{braket}
\usepackage{algorithm}
\usepackage[noend]{algpseudocode}
\makesavenoteenv{tabular}

\begin{document}
%Title
\title{Towards a robust data processing framework for telecommunications data}
\subtitle{[Extended Abstract]}

%Author(s)
\numberofauthors{1}
\author{
\alignauthor
Julia Hermann\\
       \affaddr{University of Trento}\\
       \email{julia.hermann@studenti.unitn.it}
}

%Abstract
\maketitle
%Date
\date{22 June 2018}
\begin{abstract}
Large amount of data captured by the telecommunications industry stakeholders needs to be efficiently processed. We propose a data processing framework that is flexible and can run on local machines, small clusters or even cloud computing platforms, such as Amazon Web Services. The core of this architecture is the open-source big data processing tool, Apache Spark using its Python API. We compare different storage formats and implementations for a simple problem and the most common dataframe operations using Spark.
\end{abstract}
\keywords{big data, Spark, geo-positioning, GPS, CDR}
%Main parts
\section{Introduction}\subfile{intro}
\section{Problem Statement}\subfile{problem_statement.tex}
\section{Literature Review}\subfile{related_work.tex}
\section{Solution}\subfile{solution.tex}
\section{Tools and Frameworks}\subfile{tools.tex}
\section{Experimental Evaluation}\subfile{experiments.tex}
\section{Conclusion}\subfile{conclusion.tex}

\bibliographystyle{abbrv}
\bibliography{refs}
\end{document}
