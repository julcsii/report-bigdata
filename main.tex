\documentclass{sig-alternate-05-2015}
%Hide ISBN from bottom
\makeatletter
\def\@copyrightspace{\relax}
\makeatother

\usepackage{subfiles}
\usepackage{hyperref}
\usepackage{footnote}
\makesavenoteenv{tabular}

\begin{document}
%Title
\title{CDR based geo-positioning of anonymous user in Berlin area}
\subtitle{[Extended Abstract]}

%Author(s)
\numberofauthors{1}
\author{
\alignauthor
Julia Hermann\\
       \affaddr{University of Trento}\\
       \email{julia.hermann@studenti.unitn.it}
}

%Abstract
\maketitle
%Date
\date{28 February 2018}
\begin{abstract}
This paper attempts to validate the usage of localized Call Detail Records (CDR) data for customer positioning by comparing them to GPS data, that is considered as ground-truth. The CDR events are localized with two different types of cell map data-sets, an open-source and a proprietary one. The data processing architecture relies on standalone Spark cluster running in Docker container. The analysis shows that a customer can be positioned given their CDR data with around 500 meters accuracy, depending on individual mobile usage habits, network characteristics and geographical position.

\end{abstract}
\keywords{big data, Spark, geo-positioning, GPS, CDR}
%Main parts
\section{Introduction}\subfile{intro}
\section{Problem Statement}\subfile{problem_statement.tex}
\section{Literature Review}\subfile{related_work.tex}
\section{Solution}\subfile{solution.tex}
\section{Implementation}\subfile{implementation.tex}
\section{Analysis}\subfile{analysis.tex}
\section{Experiments}\subfile{experiments.tex}
\section{Conclusion}\subfile{conclusion.tex}
%\clearpage
%References
\bibliographystyle{abbrv}
\bibliography{refs}
\end{document}
