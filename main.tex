\documentclass{sig-alternate-05-2015}
%Hide ISBN from bottom
\makeatletter
\def\@copyrightspace{\relax}
\makeatother

\usepackage{subfiles}
\usepackage{nomencl}
\makenomenclature
\usepackage{hyperref}
\usepackage{footnote}
\usepackage{braket}
\usepackage{algorithm}
\newtheorem{definition}{Definition}
\usepackage[noend]{algpseudocode}
\makesavenoteenv{tabular}

\begin{document}
%Title
\title{Towards a flexible big data processing framework for mobile telecommunications data}
\subtitle{[Extended Abstract]}

%Author(s)
\numberofauthors{1}
\author{
\alignauthor
Julia Hermann\\
       \affaddr{University of Trento}\\
       \email{julia.hermann@studenti.unitn.it}
}

%Abstract
\maketitle
%Date
\date{22 June 2018}
\begin{abstract}
Large amount of data captured by the telecommunications service providers needs to be efficiently processed. We propose a data processing framework that is flexible and can run on local machines, small clusters or even cloud computing platforms, such as Amazon Web Services. The core of this architecture is the open-source big data processing tool, Apache Spark using its Python API. We compare different implementations for a Euclidean trajectory distance calculation and different storage formats for the most common data frame operators using Spark.
\end{abstract}
\keywords{big data, Spark, geo-positioning, GPS, CDR}
%Main parts
\section{Introduction}\subfile{intro}
\section{Problem Statement}\subfile{problem_statement.tex}
\section{State-of-the-art}\subfile{related_work.tex}
\section{Solution}\subfile{solution.tex}
\section{Tools and Frameworks}\subfile{tools.tex}
\section{Experimental Evaluation}\subfile{experiments.tex}
\section{Conclusion}\subfile{conclusion.tex}

\bibliographystyle{abbrv}
\bibliography{refs}
\end{document}
