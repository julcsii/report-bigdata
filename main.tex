\documentclass{sig-alternate-05-2015}
%Hide ISBN from bottom
\makeatletter
\def\@copyrightspace{\relax}
\makeatother

\usepackage{subfiles}
\usepackage{hyperref}
\usepackage{footnote}
\makesavenoteenv{tabular}

\begin{document}
%Title
\title{Mobile network based geo-positioning of anonymous user in Berlin area}
\subtitle{[Extended Abstract]}

%Author(s)
\numberofauthors{1}
\author{
\alignauthor
Julia Hermann\\
       \affaddr{University of Trento}\\
       \email{julia.hermann@studenti.unitn.it}
}

%Abstract
\maketitle
%Date
\date{16 April 2018}
\begin{abstract}
In this paper we validate the usage of localized Call Detail Records data for customer positioning by comparing them to ground-truth data. The data processing architecture relies on standalone Spark cluster running in Docker container. The analysis shows that a customer can be positioned given their CDR data with around ??? meters accuracy, depending on individual mobile usage habits, network characteristics and geographical position.

\end{abstract}
\keywords{big data, Spark, geo-positioning, GPS, CDR}
%Main parts
\section{Introduction}\subfile{intro}
\section{Problem Statement}\subfile{problem_statement.tex}
\section{Literature Review}\subfile{related_work.tex}
\section{Solution}\subfile{solution.tex}
\section{Proposed Approach}\subfile{approach.tex}
\section{Experimental Evaluation}\subfile{experiments.tex}
\section{Conclusion}\subfile{conclusion.tex}
% P: point
% T: trajectory
% M: set of trajectories
% d: distance

%\clearpage
%References
\bibliographystyle{abbrv}
\bibliography{refs}
\end{document}
